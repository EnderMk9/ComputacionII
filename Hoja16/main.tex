%%%%%%%%%%%%%%%%%%%%%%%%%%%%%%%%%%%%%%%%%
% Lachaise Assignment
% LaTeX Template
% Version 1.0 (26/6/2018)
%
% This template originates from:
% http://www.LaTeXTemplates.com
%
% Authors:
% Marion Lachaise & François Févotte
% Vel (vel@LaTeXTemplates.com)
%
% License:
% CC BY-NC-SA 3.0 (http://creativecommons.org/licenses/by-nc-sa/3.0/)
% 
%%%%%%%%%%%%%%%%%%%%%%%%%%%%%%%%%%%%%%%%%

%----------------------------------------------------------------------------------------
%	PACKAGES AND OTHER DOCUMENT CONFIGURATIONS
%----------------------------------------------------------------------------------------
\documentclass{article}

\input{structure.tex} % Include the file specifying the document structure and custom commands
\usepackage{lscape}
%----------------------------------------------------------------------------------------
%	ASSIGNMENT INFORMATION
%----------------------------------------------------------------------------------------

\title{Práctica 14} % Title of the assignment

\author{Abel Rosado Peinado - 5265\\ \texttt{abel.rosado@estudiante.uam.es}} % Author name and email address

\date{UAM --- \today} % University, school and/or department name(s) and a date

%----------------------------------------------------------------------------------------

\begin{document}

\maketitle % Print the title
\noindent Queremos encontrar la función potencial $V(r)$ entre dos superficies esféricas conductoras cargadas, tal que $V(R_1) = V_1$ y $V(R_2) = V_2$, para ello, como el potencial es simétricamente esférico, aplicamos la ecuación de Laplace en coordendas esféricas, tal que,
\[\nabla^2 V = \frac{1}{r^2} \frac{d}{dr}\left(r^2 \frac{d V}{dr}\right) = 0 \rightarrow \frac{d V}{dr} = \frac{c_1}{r^2} \rightarrow V = \frac{A}{r}+ B\]
Ahora aplicamos las condiciones de contorno
\[\left\{\begin{matrix}
	\frac{A}{R_1}+ B = V_1 \\ \frac{A}{R_2}+ B = V_2
\end{matrix}\right. \rightarrow \left\{\begin{matrix}
	A = \frac{R_1 R_2}{R_2-R_1}(V_1-V_2) \\ B = \frac{V_2 R_2 - V_1 R_1}{R_2-R_1} \phantom{---}
\end{matrix}\right.\] 
Entonces, si $R_1 = 0.05 $ m, $R_2 = 0.1$ m, $V_1 = 110$ V y $V_2 = 0$ V, entonces $A = 11 $ Vm y $B = -110$ V, y podemos observarlo representado en la figura inferior.

\begin{minipage}{8cm}
	\centering
	\includegraphics[width=7.5cm]{untitled1.png}
  \end{minipage}%
  \begin{minipage}{8cm}
	\centering
	\includegraphics[width=7.5cm]{untitled2.png}
  \end{minipage}
Para resolverlo númericamente, la ecuación de Laplace la transformamos en un sistema de ecuaciones diferenciales
\[\nabla^2 V = \frac{1}{r^2} \frac{d}{dr}\left(r^2 \frac{d V}{dr}\right) = 0 \rightarrow \left\{\begin{matrix}
	\frac{d V}{dr} = U \\ \frac{d U}{dr} = -\frac{2}{r} U 
\end{matrix}\right.\]
Al resolverlo utilizando el shooting method, obtenemos una pendiente inicial de $-4400$ Vm$^{-1}$, impreso con 16 cifras significativas, el resultado exacto, por lo que si ahora integramos la ecuación con RK, las diferencias que veamos con la solución analítica serán unicamente las propias de la integración con RK, como se observa en la figura anterior.
\end{document}
