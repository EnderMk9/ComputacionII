%%%%%%%%%%%%%%%%%%%%%%%%%%%%%%%%%%%%%%%%%
% Lachaise Assignment
% LaTeX Template
% Version 1.0 (26/6/2018)
%
% This template originates from:
% http://www.LaTeXTemplates.com
%
% Authors:
% Marion Lachaise & François Févotte
% Vel (vel@LaTeXTemplates.com)
%
% License:
% CC BY-NC-SA 3.0 (http://creativecommons.org/licenses/by-nc-sa/3.0/)
% 
%%%%%%%%%%%%%%%%%%%%%%%%%%%%%%%%%%%%%%%%%

%----------------------------------------------------------------------------------------
%	PACKAGES AND OTHER DOCUMENT CONFIGURATIONS
%----------------------------------------------------------------------------------------
\documentclass{article}

\input{structure.tex} % Include the file specifying the document structure and custom commands
\usepackage{lscape}
%----------------------------------------------------------------------------------------
%	ASSIGNMENT INFORMATION
%----------------------------------------------------------------------------------------

\title{Práctica 14} % Title of the assignment

\author{Abel Rosado Peinado - 5265\\ \texttt{abel.rosado@estudiante.uam.es}} % Author name and email address

\date{UAM --- \today} % University, school and/or department name(s) and a date

%----------------------------------------------------------------------------------------

\begin{document}

\maketitle % Print the title
\noindent Queremos resolver el siguiente sistema de ecuaciones diferenciales ordinarias
\[\left\{\begin{matrix}
	\dot{y}_1 = v_1 \phantom{---------} \\ \dot{v}_1 =-\frac{k_1}{m_1}y_1-\frac{k_2}{m_1}(y_1-y_2) \\
	\dot{y}_2 = v_2 \phantom{---------} \\ \dot{v}_2 =\frac{k_2}{m_2}(y_1-y_2)  \phantom{-----}
\end{matrix}\right.\]
Las soluciones usando el método de Runge-Kutta de cuarto orden se observan en la figura siguiente y animados en los vídeos adjuntos.

\begin{minipage}{8cm}
	\centering
	\includegraphics[width=7.5cm]{untitled.png}
  \end{minipage}%
  \begin{minipage}{8cm}
	\centering
	\includegraphics[width=7.5cm]{untitled1.png}
  \end{minipage}

Para obtener los nodos normales tenemos la siguiente ecuación dervivada de la de arriba
\[\left(\begin{matrix}
	\ddot{y}_1 \\ \ddot{y}_2
\end{matrix}\right) = \left(\begin{matrix}
	-\frac{k_1}{m_1} -\frac{k_2}{m_1} & \frac{k_2}{m_1}\\
	\frac{k_2}{m_2} & -\frac{k_2}{m_2}
\end{matrix}\right) \left(\begin{matrix}
	y_1 \\ y_2
\end{matrix}\right) = \left(\begin{matrix}
	-3 & 1.75\\
	1 & -1
\end{matrix}\right) \left(\begin{matrix}
	y_1 \\ y_2
\end{matrix}\right) \]
Podríamos diagonalizarla, pero no por el método de Jacobi, puesto que no es diagonal, si lo hacemos obtenemos que los autovalores y autovectores son
\[\lambda = \frac{-4 \pm \sqrt{11}}{2} \ \ \ \ \ v = \left(\begin{matrix} -2 \pm \sqrt{11} \\ 2
\end{matrix}\right) \]
\end{document}
