%%%%%%%%%%%%%%%%%%%%%%%%%%%%%%%%%%%%%%%%%
% Lachaise Assignment
% LaTeX Template
% Version 1.0 (26/6/2018)
%
% This template originates from:
% http://www.LaTeXTemplates.com
%
% Authors:
% Marion Lachaise & François Févotte
% Vel (vel@LaTeXTemplates.com)
%
% License:
% CC BY-NC-SA 3.0 (http://creativecommons.org/licenses/by-nc-sa/3.0/)
% 
%%%%%%%%%%%%%%%%%%%%%%%%%%%%%%%%%%%%%%%%%

%----------------------------------------------------------------------------------------
%	PACKAGES AND OTHER DOCUMENT CONFIGURATIONS
%----------------------------------------------------------------------------------------

\documentclass{article}

\input{structure.tex} % Include the file specifying the document structure and custom commands

%----------------------------------------------------------------------------------------
%	ASSIGNMENT INFORMATION
%----------------------------------------------------------------------------------------

\title{Examen de Computación II} % Title of the assignment

\author{Abel Rosado Peinado - 5265\\ \texttt{abel.rosado@estudiante.uam.es}} % Author name and email address

\date{UAM --- \today} % University, school and/or department name(s) and a date

%----------------------------------------------------------------------------------------

\begin{document}

\maketitle % Print the title

\section*{Ejercicio 1}
\includegraphics[width=15.5cm]{ln.png}

He estimado el numero de subintervalos necesarios de dos formas, la primera, aprovechando que el error de la integral usando la Regla de Simpson es 
\[e \leq M\frac{(b-a)^5}{180 n^4} = M\frac{(x-1)^5}{180 n^4} \ \ \ \ n \leq \left( M \frac{(x-1)^5}{180 e} \right)^{1/4} \ \  \ \ |f^{4)}(t)| \leq M \ \forall t \in [1,x]\]
Estudiando la cuarta derivada de $1/x$, $24/x^4$, vemos que el máximo de esa función en el intervalo de integración es $24$, entonces para cada x, calculamos su n correspondiente e integramos, y obtenemos la primera tabla que se observa arriba.

Por otro lado, he iterado la integral y he tomado el error como la resta de los resultados de dos iteraciones consecutivas, y cuando este fuese menor que la tolerancia, paraba la iteración, así obtenemos los valores de la segunda tabla.

Observamos en ambos casos, el error con respecto a la función calculada por la función $\log$ de c++ es menor que el deseado, y observamos que cuanto mayor es x, menor es el error, que puede deberse a que cada vez la porción de área que añadimos es más pequeña.

\newpage
\section*{Ejercicio 2}
\includegraphics[width=7.5cm]{x.png}
\includegraphics[width=7.5cm]{y.png}

\includegraphics[width=7.5cm]{xy.png}
\includegraphics[width=7.5cm]{theta.png}

En las dos primeras figuras observamos $x(t)$ e $y(t)$, en la tercera observamos $x(y)$, y en la última esta representado el ángángulo $\arcsin(y/x)$ en función del número de periodos $t/P$ para 3 periodos, donde $P\approx9$\dots

Para obtener un error menor que la precisión, he iterado RK4 y en cada iteración he comparado la diferencia entre $x(t_f)$ e $y(t_f)$ para dos iteraciones sucesivas, y cuando la suma de ambas fuese menor que la tolerancia, paran las iteraciones, he obtenido que $\Delta t \approx 0.003$ s. 

Integrando la figura 4, y dividiendo por el número de periodos, obtenemos que aproximadamente el ángulo recorrido durante un perperiodo es $\theta_m=0.00073$, por lo tanto $2\pi/\theta_m \approx 8550$ periodos, lo que se traduce a unas $21.37$ horas para dar una vuelta completa.
\end{document} 
