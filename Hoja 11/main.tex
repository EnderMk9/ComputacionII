%%%%%%%%%%%%%%%%%%%%%%%%%%%%%%%%%%%%%%%%%
% Lachaise Assignment
% LaTeX Template
% Version 1.0 (26/6/2018)
%
% This template originates from:
% http://www.LaTeXTemplates.com
%
% Authors:
% Marion Lachaise & François Févotte
% Vel (vel@LaTeXTemplates.com)
%
% License:
% CC BY-NC-SA 3.0 (http://creativecommons.org/licenses/by-nc-sa/3.0/)
% 
%%%%%%%%%%%%%%%%%%%%%%%%%%%%%%%%%%%%%%%%%

%----------------------------------------------------------------------------------------
%	PACKAGES AND OTHER DOCUMENT CONFIGURATIONS
%----------------------------------------------------------------------------------------

\documentclass{article}

\input{structure.tex} % Include the file specifying the document structure and custom commands

%----------------------------------------------------------------------------------------
%	ASSIGNMENT INFORMATION
%----------------------------------------------------------------------------------------

\title{Práctica 11, Diagonalización} % Title of the assignment

\author{Abel Rosado\\ \texttt{abel.rosado@estudiante.uam.es}} % Author name and email address

\date{UAM --- \today} % University, school and/or department name(s) and a date

%----------------------------------------------------------------------------------------

\begin{document}

\maketitle % Print the title


Tendremos un bloque de viviendas formado por 4 plantas, las cuales sufren un impulso a causa de un huracán, un terremoto u otra catástrofe natural de índole similar.

Consideramos que las paredes son muelles sin masa que conectan los suelos, de tal forma que las ecuaciónes del movimiento para las posiciones horizontales de los suelos son
\[m_i \ddot{x}_i+ k_i (x_i-x_{i-1})-k_{i+1}(x_{i+1}-x_i) = 0\]
Si consideramos que que todas las masas y las constantes de los muelles son iguales tenemos el siguiente sistema de ecuaciones diferenciales lineal
\[\left\{\begin{matrix}m \ddot{x}_1 +2 k x_1 -k x_2 = 0 \phantom{---} \\
m \ddot{x}_2 +2 k x_2 -k x_1 - k x_3 = 0 \\
m \ddot{x}_3 +2 k x_3 -k x_2 - k x_4 = 0 \\
m \ddot{x}_4 +k (x_4-x_3) = 0 \phantom{---} \end{matrix}\right.\]
Que podemos expresar mediante la siguiente ecuación vectorial, donde $\mathbf{I}_4$ es la identidad
\[\mathbf{M} \ddot{\mathbf{x}} + \mathbf{K} \mathbf{x} = 0\]
\[\mathbf{x} = \left(\begin{matrix}
	x_1 \\ x_2 \\ x_3 \\ x_4
\end{matrix}\right)\ \ \ \ \ \ \ddot{\mathbf{x}} = \left(\begin{matrix}
	\ddot{x}_1 \\ \ddot{x}_2 \\ \ddot{x}_3 \\ \ddot{x}_4
\end{matrix}\right)\ \ \ \ \ \ \mathbf{M} = m \mathbf{I}_4 \ \ \ \ \ \ \mathbf{K} = k\left(\begin{matrix}
	2 && -1 &&  0 &&  0 \\
   -1 &&  2 && -1 &&  0 \\
	0 && -1 &&  2 && -1 \\
	0 &&  0 && -1 && 1
\end{matrix}\right)\]
Que podemos reescribir como
\[\ddot{\mathbf{x}} + \mathbf{C} \mathbf{x} = 0 \ \ \ \ \  \mathbf{C} =\frac{1}{m}\mathbf{K}\]
De tal forma que, si $m=4000$ kg y $k = 5000$ N/m, entonces
\[\mathbf{C} = \left(\begin{matrix}
	2.5 && -1.25 &&  0 &&  0 \\
   -1.25 &&  2.5 && -1.25 &&  0 \\
	0 && -1.25 &&  2.5 && -1.25 \\
	0 &&  0 && -1.25 && 1.25
\end{matrix}\right)\]
Si ahora diagonalizamos $\mathbf{C}$, usando el método iterativo de Jacobi, obtenemos un resutado $\mathbf{C} = \mathbf{U}\mathbf{D}\mathbf{U}^T$, al revés, $\mathbf{U}^T \mathbf{C}\mathbf{U} = \mathbf{D}$ , donde $\mathbf{U}^T = \mathbf{U}^{-1}$, tal que
\[\mathbf{D} \approx \left(\begin{matrix}
	1.25 && 0 &&  0 &&  0 \\
    0 &&  4.415 && 0&&  0 \\
	0 && 0 &&  2.934 && 0 \\
	0 &&  0 && 0&& 0.151
\end{matrix}\right) \ \ \ \ \ \mathbf{U} \approx \left(\begin{matrix}
	\sqrt{3}/3 && -0.428 &&  -0.657 &&  0.228 \\
    \sqrt{3}/3 &&  0.657 &&  0.228  &&  0.428 \\
	0 && -\sqrt{3}/3 &&  \sqrt{3}/3 && \sqrt{3}/3 \\
	-\sqrt{3}/3 &&  0.228 && -0.428&& 0.657
\end{matrix}\right)\]
Tenemos que $\mathbf{U}$ es la matriz de cambio de base desde la base de autovectores hasta la base original del problema, tal que $\mathbf{U} \mathbf{x}' = \mathbf{x}$ y $\mathbf{U} \ddot{\mathbf{x'}} = \ddot{\mathbf{x}}'$, y al revés, $ \mathbf{x}' =\mathbf{U}^T \mathbf{x}$ y $\ddot{\mathbf{x'}} = \mathbf{U}^T \ddot{\mathbf{x}}'$, entonces multiplicando a la izquierda por $\mathbf{U}^T$ en la ecuación diferencial anterior y sustituyendo $\mathbf{x}$ llegamos a 
\[\mathbf{U}^T\ddot{\mathbf{x}} + \mathbf{U}^T\mathbf{C} \mathbf{U}\mathbf{x}' = 0 \implies \ddot{\mathbf{x}}' + \mathbf{D} \mathbf{x}'=0\]
Entonces ahora tenemos un sistema de ecuaciones diferenciales \textbf{no} acopladas, de la forma siguiente, donde $\lambda_i$ es el autovalor asociado
\[\ddot{x}'_i + \lambda_i x'_i =0\]
Estas escuaciones diferenciales ordinarias tienen una solución analítica de la forma
\[x'_i(t) = a_i \cos\omega_i t + b_i \sin\omega_i t \ \ \ \ \ \ \omega_i^2 = \lambda_i\]
Que podemos expresar de nuevo como un vector, y multiplicar por $\mathbf{U}$ para obtener de nuevo $\mathbf{x}$, tal que
\[\mathbf{x}'(t) = \left(\begin{matrix}
	a_1 \cos\omega_1 t + b_1 \sin\omega_1 t \\
	a_2 \cos\omega_2 t + b_2 \sin\omega_2 t \\
	a_3 \cos\omega_3 t + b_3 \sin\omega_3 t \\
	a_4 \cos\omega_4 t + b_4 \sin\omega_4 t
\end{matrix}\right) \ \ \ \ \ \mathbf{x} = \mathbf{U} \mathbf{x}' \ \ \ \ \  [a_i] = \mathbf{U}^T \mathbf{x}(0) \ \ \ \ \  [\omega_i b_i] = \mathbf{U}^T \dot{\mathbf{x}}(0)\]
De esta forma tenemos un sistema de osciladores acoplados, y como todos los autovalores son positivos, la resolución de la ecuación diferencial nos devuelve soluciones oscilantes estables con las raices de los autovalores como frecuencias. Si uno o más de los autovalores hubiera sido negativo, una de las soluciones hubiera sido una exponencial y la evolución temporal no sería estable en la mayor parte de las condiciones iniciales.

Las frecuencias, en Hz, son $f_1 = 0.178$, $f_2 = 0.334$, $f_3 = 0.273$ y $f_4 = 0.0618$, oscilaciones perceptibles a simple vista para tiempos relativamente cortos del orden de segundos.
\end{document}
